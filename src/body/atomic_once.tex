\setaachaptext[51]{an appendix presenting atomic-free atomics}
\chapter{Atomic once}

Atomic set a value, safe for any participating threads, guarantees to set once.
Only based on read/write of shared memory.

%\vspace{2\baselineskip}
%\begin{lstcode}[caption={Outline of a C++ implementation},label=lst:atomic:cpp]
%class EnterOnce {
%public:
%	EnterOnce(bool set=false){
%		if(set) m_entry=true;
%			else Reset();
%	}
%	bool IsTouched() const { return (bool)m_entry; }
%	operator bool() const {  return IsTouched(); }
%protected:
%	void Reset() {			 memset(&m_entry,0,sizeof(m_entry)); }
%
%#ifdef AVOID_ATOMIC // consider atomic operations as extremely expensive
%	typedef uintptr_t id_t;
%	typedef uint8_t sid_t;
%	static const sid_t ID_IN_CS=(sid_t)-1;
%	
%	bool Entry(){
%		id_t id=GetID();
%		sid_t sid;
%
%		// lock step: n to {1,2}
%		m_entry.step1=id;
%		if(unlikely(m_entry.step2!=0))
%			return false;
%		m_entry.step2=id;
%
%		if(likely(m_entry.step1==id)){
%			sid=1;
%			m_entry.step3=sid;
%		}else{
%			sid=2;
%			if(m_entry.step3.cas(0,sid)!=0)
%				return false;
%		}
%
%		// lock m_entry.step: {1,2} to 1
%wait_step4:
%		switch(m_entry.step4){
%		case ID_IN_CS:
%			return false;
%		case 0:
%			m_entry.step4=sid;
%			if(likely(m_entry.step3==sid)){
%				m_entry.step4=ID_IN_CS;
%				return true;
%			}else{
%				m_entry.step4=0;
%				return false;
%			}
%		default:
%			goto wait_step4;
%		}
%	}
%
%private:
%	id_t GetID(){
%		// generate some thread-unique value
%		int dummy=0;
%		id_t id=(id_t)&dummy;
%		return id;
%	}
%	
%	struct {
%		atomic_t<id_t> step1,step2;
%		atomic_t<sid_t> step3,step4;
%		operator bool() const { return step1!=0; }
%		void operator=(const bool rhs){ step1=1; }
%	} m_entry;
%
%#else // consider atomic operations as cheap
%	bool Entry(){
%		return __sync_bool_compare_and_swap(&m_entry,false,true);
%	}
%private:
%	bool m_entry;
%#endif // AVOID_ATOMIC
%};
%\end{lstcode}

\vspace{2\baselineskip}
\begin{lstcode}[language=Promela,caption={Promela implementation for verification},label=lst:atomic:promela]
#define PROCS	6
#define MAX_ID	(PROCS+1)

byte step1=0;
byte step2=0;
byte step3=0;
byte step4=0;
byte in_cs=0;
byte proc_done=0;

proctype worker(byte id) {

	////////////////////////////////////
	// Enter lock

    // lock step: n to {1,2}
    step1=id;
    if  
    :: step2==0 -> step2=id
    :: else -> goto done
    fi; 
    if
    :: step1==id -> step3=id
    :: else ->  
		if  
		:: step3==0 -> step3=id
		:: else -> goto done
		fi  
    fi; 

    // lock step: {1,2} to 1
    if
    :: step4==MAX_ID -> goto done
    :: step4==0 -> step4=id
    fi; 
    if  
    :: step3==id -> step4=MAX_ID
    :: else -> step4=0; goto done
    fi; 
	
	////////////////////////////////////
	// Critical section

    printf("%d in cs\n",id);
    in_cs++;
    assert(in_cs==1);
done:
    proc_done++
}

init { 
	byte procs=0;
    atomic{
        do  
        :: procs<PROCS -> procs++; run worker(procs)
        :: procs>=1 && procs<PROCS -> break
        :: procs>=PROCS -> break;
        od; 
    }
    in_cs==1&&proc_done==procs;
}
\end{lstcode}

